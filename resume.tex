% !TEX program = xelatex

\documentclass{resume}
%\usepackage{zh_CN-Adobefonts_external} % Simplified Chinese Support using external fonts (./fonts/zh_CN-Adobe/)
%\usepackage{zh_CN-Adobefonts_internal} % Simplified Chinese Support using system fonts

\begin{document}
\pagenumbering{gobble} % suppress displaying page number

\name{Wei Shen}

\basicInfo{
  \email{shenwei0824@gmail.com} \textperiodcentered\ 
  \phone{(+86) 17788759927} \textperiodcentered\ 
  \linkedin[Wei Shen]{https://www.linkedin.com/in/wei-shen-1a1197162/}}

\section{\faGraduationCap\ Education}
\datedsubsection{\textbf{University of Chinese Academy of Sciences} Beijing, China}{Aug. 2014 -- Jun. 2017}
\textit{Master} Nuclear Technology and Applications 
\datedsubsection{\textbf{University of Science and Technology of China} Hefei, China}{Aug. 2010 -- Jun. 2014}
\textit{B.S.} Nuclear Engineering and Technology

\section{\faUsers\ Experience}
\datedsubsection{\textbf{Tencent} Guangzhou, China}{Oct. 2018 -- Present}
\role{Software Engineer, Framework Development, Distributed System}{}

WQueue, WeChat distributed queue (like Kafka/Pulsar)
\begin{itemize}
  \item Developed enqueue/dequeue framework based on libco and svrkit, the coroutine library and RPC framework of WeChat
  \item Developed data storage logic based on WFS, the distributed filesystem of WeChat (like HDFS)
  \item Developed consumer schedule logic based on Chubby, the distributed lock service of WeChat (like zookeeper and Google chubby)
  \item Developed a simple management web page, fronend based on Vue.js and backend with gin
  \item Developed some Java adapters to run MapReduce/Flink with WQueue
  \item WQueue is now widely used in WeChat for micellaneous purposes
\end{itemize}

Migration of WeChat services to Kubernetes
\begin{itemize}
  \item Participated in designing the archtecture of WeChat services on K8s
  \item Adapted old components of WeChat services to be run on K8s
  \item Developed neccessary procedures to help WFS to run on K8s
\end{itemize}

Svrkit, the RPC framework of WeChat
\begin{itemize}
  \item Developed features such as gateway logic, route method by tag, monitor system based on multi-level hash table in shared memory
\end{itemize}

\datedsubsection{\textbf{Tencent} Shenzhen, China}{Jul. 2017 -- Oct. 2018}
\role{Software Engineer, Business Backend Development}{}
The business backend logic of QQWallet/QQCoupon
\begin{itemize}
  \item Developed features for QQWallet/QQCoupon, such as QQWallet Daily Task, QQCoupon Spring Festival Activity
\end{itemize}

Feeds System for a standalone app (like Instagram, terminated during internal test)
\begin{itemize}
  \item Developed features like publishment/management/comments in the backend for an app
\end{itemize}

%\datedsubsection{\textbf{Sogou(Intern)} Beijing, China}{May. 2016 -- Aug. 2016}
%\role{Software Engineer, Data Engineer}{}
%Business log analysis system
%\begin{itemize}
%  \item Developed Hadoop procedure to process log reported by different app, and stored result in HBase
%  \item Developed web page for presentation of the result
%\end{itemize}


% Reference Test
%\datedsubsection{\textbf{Paper Title\cite{zaharia2012resilient}}}{May. 2015}
%An xxx optimized for xxx\cite{verma2015large}
%\begin{itemize}
%  \item main contribution
%\end{itemize}

\section{\faCogs\ Skills}
\begin{itemize}[parsep=0.5ex]
  \item Proficient in C++/Linux/Network/Multithreaded programming
  \item Some knowledge of Java/JavaScript/Paxos/Raft/LevelDB/Docker/K8s
\end{itemize}

%\section{\faHeartO\ Honors and Awards}
%\datedline{Best(top 10\%) of performance assessment in Tencent every half year}{Jul. 2019}
%\datedline{Best(top 10\%) of performance assessment in Tencent every half year}{Jul. 2018}


%% Reference
%\newpage
%\bibliographystyle{IEEETran}
%\bibliography{mycite}
\end{document}
