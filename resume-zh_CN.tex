% !TEX TS-program = xelatex
% !TEX encoding = UTF-8 Unicode
% !Mode:: "TeX:UTF-8"

\documentclass{resume}
\usepackage{zh_CN-Adobefonts_external} % Simplified Chinese Support using external fonts (./fonts/zh_CN-Adobe/)
% \usepackage{NotoSansSC_external}
% \usepackage{NotoSerifCJKsc_external}
% \usepackage{zh_CN-Adobefonts_internal} % Simplified Chinese Support using system fonts
\usepackage{linespacing_fix} % disable extra space before next section
\usepackage{cite}

\begin{document}
\pagenumbering{gobble} % suppress displaying page number

\name{沈炜}

\basicInfo{
  \email{shenwei0824@gmail.com} \textperiodcentered\ 
  \phone{(+86) 17788759927} \textperiodcentered\ 
  \linkedin[Wei Shen]{https://www.linkedin.com/in/wei-shen-1a1197162/}}
 
\section{\faGraduationCap\ 教育背景}
\datedsubsection{\textbf{中国科学院大学}, 北京}{2014 -- 2017}
\textit{硕士}\ 核技术及应用
\datedsubsection{\textbf{中国科学技术大学}, 安徽}{2010 -- 2014}
\textit{学士}\ 核工程与核技术

\section{\faUsers\ 项目经历}
\datedsubsection{\textbf{腾讯} 广州}{2018年10月 -- 至今}
\role{后台开发,框架开发,分布式系统}{}
微信分布式队列(类kafka/pulsar)开发与维护
\begin{itemize}
  \item 基于微信协程库 libco 和 RPC 框架实现高性能入队逻辑模块和出队消费框架
  \item 使用微信分布式文件系统(类HDFS)落地队列数据,根据队列需求定制化系统功能
  \item 基于微信分布式锁服务(类Chubby)实现自动消费调度和消费偏移管理,实现At Least Once消费
  \item 目前该队列已服务于微信内部实时日志数据转发、跨城最终一致存储、离线计算平台等项目
\end{itemize}

微信后台 RPC 框架开发与维护
\begin{itemize}
  \item 日常业务特性开发,比如新增网关转发/定制化路由等功能
  \item 运营系统特性开发,比如基于共享内存多阶hash表的打点上报系统
\end{itemize}

\datedsubsection{\textbf{腾讯} 深圳}{2017年7月 -- 2018年10月}
\role{后台开发,业务开发}{}
\begin{onehalfspacing}
QQ钱包/QQ卡券后台业务逻辑开发
\begin{itemize}
  \item 基于RPC/KV/MySQL等内外部组件开发业务后台逻辑,如钱包任务中心后台、卡券首页后台
\end{itemize}

独立内容APP后台FEEDS流系统搭建
\begin{itemize}
  \item 基于腾讯云COS和内外部组件实现作品发布/管理/评论等功能,构建完整FEEDS系统
\end{itemize}
\end{onehalfspacing}
%
\datedsubsection{\textbf{搜狗(实习)} 北京}{2016年5月 -- 2016年8月}
\role{数据开发,网页开发}{}
\begin{onehalfspacing}
搜狗业务日志分析系统
\begin{itemize}
  \item 根据json定义的日志格式,使用Hadoop处理多种业务的日志,计算需要的结果存入HBase
  \item 设计/开发网页,根据选择的查询条件将HBase中的数据通过表格、图像等方式进行展示
\end{itemize}
\end{onehalfspacing}

% Reference Test
%\datedsubsection{\textbf{Paper Title\cite{zaharia2012resilient}}}{May. 2015}
%An xxx optimized for xxx\cite{verma2015large}
%\begin{itemize}
%  \item main contribution
%\end{itemize}

\section{\faCogs\ IT 技能}
% increase linespacing [parsep=0.5ex]
\begin{itemize}[parsep=0.5ex]
  \item 熟悉基本数据结果与算法,了解计算机基本运行原理
  \item 熟练掌握 C++ 编程,熟悉 Linux 环境与网络协议
  \item 熟悉 paxos/raft 等一致性协议,对于 leveldb/分布式存储有一定的了解
\end{itemize}

\section{\faHeartO\ 获奖情况}
\datedline{\textit{ 腾讯五星优秀员工(top 10\%) }}{ 2018 年 7 月 }
\datedline{\textit{ 腾讯五星优秀员工(top 10\%) }}{ 2019 年 7 月 }

%\section{\faInfo\ 其他}
%% increase linespacing [parsep=0.5ex]
%\begin{itemize}[parsep=0.5ex]
%  \item 技术博客: http://blog.yours.me
%  \item GitHub: https://github.com/username
%  \item 语言: 英语 - 熟练(TOEFL xxx)
%\end{itemize}

%% Reference
%\newpage
%\bibliographystyle{IEEETran}
%\bibliography{mycite}
\end{document}
